% This is sigproc-sp.tex -FILE FOR V2.6SP OF ACM_PROC_ARTICLE-SP.CLS
% OCTOBER 2002
%
% It is an example file showing how to use the 'acm_proc_article-sp.cls' V2.6SP
% LaTeX2e document class file for Conference Proceedings submissions.
% ----------------------------------------------------------------------------------------------------------------
% This .tex file (and associated .cls V2.6SP) *DOES NOT* produce:
%       1) The Permission Statement
%       2) The Conference (location) Info information
%       3) The Copyright Line with ACM data
%       4) Page numbering
%
%  However, both the CopyrightYear (default to 2002) and the ACM Copyright Data
% (default to X-XXXXX-XX-X/XX/XX) can still be over-ridden by whatever the author
% inserts into the source .tex file.
% e.g.
% \CopyrightYear{2003} will cause 2003 to appear in the copyright line.
% \crdata{0-12345-67-8/90/12} will cause 0-12345-67-8/90/12 to appear in the copyright line.
%
% ---------------------------------------------------------------------------------------------------------------
% It is an example which *does* use the .bib file (from which the .bbl file
% is produced).
% REMEMBER HOWEVER: After having produced the .bbl file,
% and prior to final submission,
% you need to 'insert'  your .bbl file into your source .tex file so as to provide
% ONE 'self-contained' source file.
%
% Questions regarding SIGS should be sent to
% Adrienne Griscti ---> griscti@acm.org
%
% Questions/suggestions regarding the guidelines, .tex and .cls files, etc. to
% Gerald Murray ---> murray@acm.org
%
% For tracking purposes - this is V2.6SP - OCTOBER 2002


\documentclass[12pt]{article}

\setlength{\oddsidemargin}{0in}
\setlength{\evensidemargin}{0in}
\setlength{\topmargin}{0in}
\setlength{\headheight}{0in}
\setlength{\headsep}{0in}
\setlength{\textwidth}{6in}
\setlength{\textheight}{9in}
\setlength{\parindent}{0in} 

\usepackage{graphicx} %For jpg figure inclusion
\usepackage{times} %For typeface
\usepackage{epsfig}
\usepackage{color} %For Comments
\usepackage[all]{xy}
\usepackage{float}
\usepackage{subfigure} 

%% Elena's favorite green (thanks, Fernando!)
\definecolor{ForestGreen}{RGB}{34,139,34}
% Uncomment this if you want to show work-in-progress comments
\newcommand{\comment}[1]{{\bf \tt  {#1}}}
% Uncomment this if you don't want to show comments
%\newcommand{\comment}[1]{}
\newcommand{\jcomment}[1]{\textcolor{ForestGreen}{\comment{Jeff: {#1}}}}
\newcommand{\emcomment}[1]{\textcolor{blue}{\comment{Elena: {#1}}}}
\newcommand{\scomment}[1]{\textcolor{red}{\comment{Seth: {#1}}}}
\newcommand{\todo}[1]{\textcolor{green}{\comment{To Do: {#1}}}}

%%%%%%%%% Abbreviations of commonly used Java types %%%%%%%%%%%%
\newcommand{\AL}{{\tt ArrayList}}
\newcommand{\ALI}{{\tt ArrayListInteger}}
\newcommand{\ALIP}{{\tt ArrayList<Integer>}} % Array List with Integer parameter
\newcommand{\ALS}{{\tt ArrayListString}}
\newcommand{\ALSP}{{\tt ArrayList<String>}}
\newcommand{\ALN}{{\tt ArrayListNumber}}
\newcommand{\ALT}{{\tt ArrayList<T>}}
\newcommand{\List}{{\tt List}} %%%%%%%% \L is already defined
\newcommand{\LI}{{\tt ListInteger}}
\newcommand{\LIP}{{\tt List<Integer>}}
\newcommand{\LT}{{\tt List<T>}}
\newcommand{\ABL}{{\tt AbstractList}}
\newcommand{\ABLI}{{\tt AbstractListInteger}}
\newcommand{\Obj}{{\tt Object}}
\newcommand{\Int}{{\tt Integer}}
\newcommand{\Str}{{\tt String}}
\newcommand{\Num}{{\tt Number}}
\newcommand{\LR}{{\tt ListReader}}
\newcommand{\LRI}{{\tt ListReaderInteger}}
\newcommand{\LRIP}{{\tt ListReader<Integer>}}
\newcommand{\LRT}{{\tt ListReader<T>}}
\newcommand{\CMP}{{\tt Comparable}}
\newcommand{\CMPT}{{\tt Comparable<T>}}
\newcommand{\PQ}{{\tt PriorityQueue}}
\newcommand{\OHM}{{\tt HashMap}}
\newcommand{\OHMKV}{{\tt HashMap<K,V>}}
\newcommand{\NIS}{{\tt NarrowedIS}} %%%% Note: this is not backward-compatible with past year's macros
\newcommand{\NR}{{\tt Narrowed}}
\newcommand{\Gen}{{\tt Generic}}
\newcommand{\NISC}{{\tt NarrowedISComp}}
\newcommand{\cv}{{\tt IS-contains-v}}
\newcommand{\cvi}{{\tt containsValue}}
\newcommand{\ISPI}{{\tt IS-put-inl}}
\newcommand{\ISCVI}{{\tt IS-contains-v-inl}}
\newcommand{\IICV}{{\tt II-contains-v-inl}}

\newcommand{\T}{{\tt T}} % type parameter
\newcommand{\TB}{{\tt <T>}} % type parameter with brackets
 
%%%%%%%%% R-prefixed classes and TestInteger %%%%%%%%%%%%%%%%%%%%%%%
\newcommand{\RAL}{{\tt RArrayList}}
\newcommand{\RALI}{{\tt RArrayListInteger}}
\newcommand{\RL}{{\tt RList}}
\newcommand{\TI}{{\tt TestInteger}}

%%%%%%%%% Abbreviations of commonly used methods %%%%%%%%%%%%
\newcommand{\get}{{\tt get}}
\newcommand{\set}{{\tt set}}
\newcommand{\add}{{\tt add}}
\newcommand{\getP}{{\tt get()}}
\newcommand{\setP}{{\tt set()}}
\newcommand{\addP}{{\tt add()}}
\newcommand{\equ}{{\tt equals}}
\newcommand{\tG}{{\tt testGet}}
\newcommand{\tS}{{\tt testSet}}
\newcommand{\tA}{{\tt testAdd}}

%%%%%%%%% Abbreviations of commonly used bytecode keywords %%%%%%%%%%%%
\newcommand{\invv}{{\tt invokevirtual}}
\newcommand{\invi}{{\tt invokeinterface}}
\newcommand{\cast}{{\tt checkcast}}

%%%%%%%%%%% Groups of specializations %%%%%%%%%%%%%%%
\newcommand{\Ogroup}{{\bf O}-group}
\newcommand{\ALgroup}{{\bf AL}-group}
\newcommand{\Cgroup}{{\bf C}-group}





\begin{document}
\pagestyle{plain}
%
% --- Author Metadata here ---
%\conferenceinfo{WOODSTOCK}{'97 El Paso, Texas USA}
%\setpagenumber{50}
%\CopyrightYear{2002} % Allows default copyright year (2002) to be
%over-ridden - IF NEED BE. 
%\crdata{0-12345-67-8/90/01}  % Allows default copyright data
%(X-XXXXX-XX-X/XX/XX) to be over-ridden. 
% --- End of Author Metadata ---

\title{Improving the Interoperability between Java and Clojure}
%\subtitle{[Extended Abstract \comment{DO WE NEED THIS?}]
%\titlenote{}}
%
% You need the command \numberofauthors to handle the "boxing"
% and alignment of the authors under the title, and to add
% a section for authors number 4 through n.
%
% Up to the first three authors are aligned under the title;
% use the \alignauthor commands below to handle those names
% and affiliations. Add names, affiliations, addresses for
% additional authors as the argument to \additionalauthors;
% these will be set for you without further effort on your
% part as the last section in the body of your article BEFORE
% References or any Appendices.

\author{
Stephen Adams \\
Computer Science Discipline \\
University of Minnesota Morris\\
Morris, MN 56267\\
adams601@morris.umn.edu
}

\date{}

\maketitle
\thispagestyle{empty}

\section*{\centering Abstract}



\newpage
\setcounter{page}{1}

\section{Introduction}\label{sec:intro}
	The Clojure programming language was first released in 2007. Clojure was designed with several things in mind \cite{cloj:rationale}:
	\begin{itemize}
	\item Functional Programming
	\item Concurrency
	\item LISP
	\item "Symbiotic" with an established platform 
	\end{itemize}
	These items are the main  

\section{Background}\label{sec:bg}

\section{Suggestions}\label{sec:sugg}

\section{Conclusion and Future Work}\label{sec:con}



%\balancecolumns

%
% The following two commands are all you need in the
% initial runs of your .tex file to
% produce the bibliography for the citations in your paper.
%\bibliographystyle{abbrv}
%\end{thebibliography}

%\bibliography{generic_types}  
% You must have a proper ".bib" file
%  and remember to run:
% latex bibtex latex latex
% to resolve all references
%
% ACM needs 'a single self-contained file'!
%
\bibliographystyle{ACM}
\bibliography{mics2012bibliography}


% That's all folks!
\end{document}

%%%%%%%%%%%%%%%%%%%%%%%%%%%%%%%%%%%%%%%%%%%%%%%%%%%%%%%%%%%%%%%%

